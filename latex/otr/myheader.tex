%%%% Package e simili %%%%

% Migliora la gestione dei font (con PDFLaTeX)
\usepackage{microtype}

% Rientro anche nel primo capoverrso, secondo la consuetudine italiana
\usepackage{indentfirst}

% Comando \vref
\usepackage[italian]{varioref}

% vari opzioni per gli ambienti enumerate
\usepackage{enumerate}

% tabelle e didascalie
\usepackage{booktabs}
\usepackage[font=small,format=hang,labelfont={bf}]{caption}
\usepackage{subfig} % sotto-figure e sotto-tabelle

% celle multiriga nelle tabelle
\usepackage{multirow}

% Font migliori per \mathbb
\usepackage{bbm} 

% pseudo-codice
\usepackage{algorithmic} 
\usepackage{algorithm}
\floatname{algorithm}{Algoritmo}
\renewcommand{\algorithmiccomment}[1]{//#1} %commenti con //

% principalmente per l'ambiente comment
\usepackage{verbatim}

% tabelle colorate
\usepackage{colortbl}

\usepackage{array}
\usepackage{tabularx}


%%%% Teoremi e cose simili %%%%

\newtheorem{theorem}{Teorema}
\newtheorem{lemma}{Lemma}
\newtheorem{proposition}{Proposizione}
\newtheorem{corollary}{Corollario}
\newtheorem{definition}{Definizione}



%%%% Un po' di cose utili per le formule %%%%

\usepackage{mathtools}
\DeclarePairedDelimiter{\abs}{\lvert}{\rvert} %valore assoluto

% trasposta di una matrice o un vettore
\newcommand{\transpose}[1]{\ensuremath{#1^{\mathrm{T}}}} 
% vettore (in bold)
\newcommand{\vect}[1]{\ensuremath{\mathbf{#1}}}
% matrice (maiuscola in bold)
\newcommand{\matr}[1]{\ensuremath{\mathbf{\uppercase{#1}}}} 
% norma vettoriale
\DeclarePairedDelimiter{\vectnorm}{\lVert}{\rVert}
% 'e' di Nepero
\providecommand*{\eu}{\ensuremath{\mathrm{e}}}
% grafi (in bold)
\newcommand{\graph}[1]{\ensuremath{\mathbf{#1}}}
% insiemi numerici
\newcommand{\numberset}{\mathbb}
\newcommand{\N}{\numberset{N}}
\newcommand{\R}{\numberset{R}}
% definizione di insiemi
\newcommand{\setdef}[2]{\lbrace #1 \vert\ #2 \rbrace}

% forza la disposizione degli oggetti mobili al termine di una sezione
% Utile per le appendici con molte tabelle
\usepackage[section]{placeins}

% ambiente per l'introduzione dei singoli capitoli, peppe's style
\newenvironment{intro}{\sffamily}{\vspace*{2ex minus 1.5ex}}

%%%% Altro %%%%

% note a margine in rosso
\newcommand{\mnote}[1]{\marginpar{\tiny {\color{red}#1}}}

% comando TODO
\newcommand{\TODO}[1]{\begin{center}\scshape \color{red} TODO (#1)\end{center}}
\newcommand{\iTODO}[1]{\scshape \color{red} TODO (#1)} % inline TODO

%%%% Link e compagnia bella %%%%

% url e link
\usepackage{hyperref}
\hypersetup{colorlinks,
            citecolor=black,
            filecolor=black,
            linkcolor=black,
            urlcolor=black,
            pdfauthor={Giulio Visco},
			pdftitle={Modelli neurali costruttivi di tipo Reservoir Computing per domini strutturati}
}
\usepackage{bookmark}
\usepackage{url}