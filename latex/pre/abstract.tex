Il presente lavoro di tesi introduce e discute nuovi modelli di Reti Neurali Ricorsive per l'apprendimento supervisionato di trasduzioni su grafi. Due sono i maggiori contributi apportati: l'adozione di un approccio costruttivo, e l'introduzione di un meccanismo stabile di output-feedback, entrambi innovativi nell'ambito del Reservoir Computing a cui si rifanno i modelli considerati. La combinazione di una strategia costruttiva e dell'utilizzo di modelli di Reservoir Computing ha inoltre permesso la realizzazione di modelli molto efficienti dal punto di vista computazionale.

I modelli e le strategie individuate si configurano come uno strumento utile e flessibile nel trattamento di domini complessi attraverso tecniche di Machine Learning, e propongono soluzioni ad alcuni dei problemi aperti nell'ambito del Reservoir Computing.

L'analisi sperimentale svolta riguarda l'apprendimento di trasduzioni strutturali da dataset reali appartenenti all'ambito della Chemioinformatica.



\begin{comment}
\small
Il Machine Learning si propone di ampliare la classe dei problemi trattabili laddove non risultino applicabili approcci algoritmici o analitici. Rilevante è dunque in questo ambito il trattamento di domini strutturati: poiché in numerosi campi le relazioni tra i dati possono essere rappresentate sotto forma di grafi, estendere la gamma di strategie utili ad apprendere trasduzioni strutturali può infatti dare risposta a problemi esistenti e permettere lo sviluppo di nuove applicazioni.

Nel contesto del Machine Learning, il trattamento di domini strutturati attraverso il paradigma neurale trova principalmente due ostacoli: l'alto costo computazionale e la presenza di vincoli sulla classe degli input trattabili. Il Reservoir Computing propone soluzioni ad entrambi i problemi, offrendo modelli efficienti in grado di apprendere trasduzioni strutturali definite su grafi generici. 

Nel corso della tesi vengono introdotti nuovi modelli di Reti Neurali Ricorsive, nell'ambito del Reservoir Computing, per l'apprendimento supervisionato di trasduzioni su grafi. I modelli proposti si caratterizzano per l'adozione di un approccio costruttivo, innovativo nell'ambito del Reservoir Computing, che permette di individuare soluzioni ad alcuni dei problemi che caratterizzano i modelli esistenti. I modelli proposti permettono infatti la determinazione della topologia della rete attraverso un processo adattivo e la realizzazione di un meccanismo stabile di output-feedback per introdurre informazione supervisionata all'interno del processo di encoding del reservoir. La strategia costruttiva permette inoltre la realizzazione di modelli molto efficienti dal punto di vista computazionale.

L'analisi sperimentale svolta riguarda l'apprendimento di trasduzioni strutturali da dataset reali appartenenti all'ambito della Chemioinformatica.
\end{comment}