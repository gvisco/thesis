Il mio primo ringraziamento va al professor Alessio Micheli, per avermi introdotto al mondo del Machine Learning e per avermi coinvolto in un'esperienza didattica diversa e stimolante, fatta di idee e di discussioni, di ricerche e di ricerca. Conserverò un ricordo prezioso delle lezioni, delle riunioni, dei consigli e delle piacevoli chiacchierate.

Un ringraziamento particolare va anche a Claudio Gallicchio che mi ha guidato in questo lungo e non sempre facile percorso, tendendomi la mano o pungolandomi secondo necessità. Tutto quello che segue è anche merito suo, e chissà se troverò mai il modo di ripagarlo del supporto ricevuto --- tecnico e soprattutto umano --- o anche solo di farmi perdonare dei grattacapi che ha dovuto subire per vedermi arrivare in fondo a questa tesi.

Ci sarebbero altre persone da ringraziare, e non sarebbero poche: amici vicini e lontani, recenti o di vecchia data, parenti, coinquilini e case, qualche professore, qualche luogo. A molti devo un ringraziamento per i consigli, per avermi supportato o semplicemente distratto, per aver ascoltato le mie lamentele o per aver cercato di capire cosa stessi combinando. Sono troppi perché un elenco possa rendere giustizia a tutti: nessuno se ne abbia a male dunque se qui sorvolo, nella speranza di poter ringraziare ognuno come merita.

Ogni singola parola, formula, tabella o figura di questa tesi è dedicata a mia madre e Cecilia, che mi hanno supportato e sopportato più di quanto si potesse desiderare.